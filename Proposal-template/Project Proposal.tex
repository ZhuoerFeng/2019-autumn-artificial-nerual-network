\documentclass{article}

% if you need to pass options to natbib, use, e.g.:
% \PassOptionsToPackage{numbers, compress}{natbib}
% before loading nips_2016
%
% to avoid loading the natbib package, add option nonatbib:
% \usepackage[nonatbib]{nips_2016}

%\usepackage{nips_2016}

% to compile a camera-ready version, add the [final] option, e.g.:
\usepackage[final]{Project_Proposal}

\usepackage[utf8]{inputenc} % allow utf-8 input
\usepackage[T1]{fontenc}    % use 8-bit T1 fonts
\usepackage{hyperref}       % hyperlinks
\usepackage{url}            % simple URL typesetting
\usepackage{booktabs}       % professional-quality tables
\usepackage{amsfonts}       % blackboard math symbols
\usepackage{nicefrac}       % compact symbols for 1/2, etc.
\usepackage{microtype}      % microtypography
\usepackage{CJKutf8}

\begin{CJK*}{UTF8}{gbsn}
\title{《人工神经网络》大作业开题报告}


% The \author macro works with any number of authors. There are two
% commands used to separate the names and addresses of multiple
% authors: \And and \AND.
%
% Using \And between authors leaves it to LaTeX to determine where to
% break the lines. Using \AND forces a line break at that point. So,
% if LaTeX puts 3 of 4 authors names on the first line, and the last
% on the second line, try using \AND instead of \And before the third
% author name.

\author{
  黄民烈\thanks{可利用脚注提供作者的更多信息} \\
  计算机科学与技术系 \\
  清华大学 \\
  \texttt{aihuang@tsinghua.edu.cn} \\
  %% examples of more authors
  \AND
  黄斐\\
  计算机科学与技术系 \\
  清华大学 \\
  \texttt{huangfei382@163.com} \\
  %% \And
  %% Coauthor \\
  %% Affiliation \\
  %% Address \\
  %% \texttt{email} \\
}

\begin{document}

\maketitle



\section{任务定义}

\textbf{准确地}定义你要做的任务,并尝试利用数学语言去形式化你的任务。

\section{数据集}

描述你准备使用的数据集(现有的还是自己整理的?),如果是自己整理的,需要给出数据集的相关统计指标。


\section{挑战和基线}

提出该任务的难点,并叙述目前在该任务上已有的工作。

\subsection{挑战}

描述你将面临的挑战。

\subsection{基线}

你可以调研和你的工作相关的文献,并选择一些基线。列举的时候可以采取下述两种方式:(也可以采取其他的,这里仅作示例)


\begin{itemize}

\item 基线1.

\item 基线2.

\end{itemize}

\paragraph{Paragraphs}

也可利用 \verb+\paragraph+ 命令来列举。

\section{研究计划}


清楚地描述你的想法,并将其和数学语言甚至代码联系起来。可以使用脚注、图和表来描述你的想法。


\subsection{脚注}

脚注\footnote{可进行补充说明。}


\subsection{图}

\begin{figure}[h]
  \centering
  \fbox{\rule[-.5cm]{0cm}{4cm} \rule[-.5cm]{4cm}{0cm}}
  \caption{模型总览}
\end{figure}

\subsection{表}

表格~\ref{sample-table}.

\begin{table}[t]
  \caption{数据集描述}
  \label{sample-table}
  \centering
  \begin{tabular}{lll}
    \toprule
    \multicolumn{2}{c}{实体}                   \\
    \cmidrule{1-2}
    类别     & 描述     & 长度 ($\mu$m) \\
    \midrule
    Dendrite & Input terminal  & $\sim$100     \\
    Axon     & Output terminal & $\sim$10      \\
    Soma     & Cell body       & up to $10^6$  \\
    \bottomrule
  \end{tabular}
\end{table}

\section{可行性}

分析该任务的可行性。


\section*{参考文献}


\medskip

\small

[1] Alexander, J.A.\ \& Mozer, M.C.\ (1995) Template-based algorithms
for connectionist rule extraction. In G.\ Tesauro, D.S.\ Touretzky and
T.K.\ Leen (eds.), {\it Advances in Neural Information Processing
  Systems 7}, pp.\ 609--616. Cambridge, MA: MIT Press.

[2] Bower, J.M.\ \& Beeman, D.\ (1995) {\it The Book of GENESIS:
  Exploring Realistic Neural Models with the GEneral NEural SImulation
  System.}  New York: TELOS/Springer--Verlag.

[3] Hasselmo, M.E., Schnell, E.\ \& Barkai, E.\ (1995) Dynamics of
learning and recall at excitatory recurrent synapses and cholinergic
modulation in rat hippocampal region CA3. {\it Journal of
  Neuroscience} {\bf 15}(7):5249-5262.

\end{document}

\end{CJK*}